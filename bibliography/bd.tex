\documentclass[a4paper,11pt]{article}
\usepackage{fullpage,color,xcolor,natbib}
\color{black}
\author{Li Ying}
\title{A Literature Survey on Big Data}
\bibliographystyle{unsrt}

\begin{document}
\maketitle


\subsection*{Big Data: A Survey}
{\color{cyan} {\color{magenta} Cited by: 12}

In this paper \cite{bdsurvey}, 
we review the background and state-of-the-art of big data. 

We first introduce the general background of big data and review related technologies,
such as could computing, Internet of Things, data centers, and Hadoop. 

We then focus on the four phases of the value chain of big data, i.e., 
{\color{red} \em data generation, data acquisition, data storage, and data analysis}. 

For each phase, we 
introduce the general background, 
discuss the technical challenges, and 
review the latest advances. 

We finally examine the several representative applications of big data, 
including
{\color{red} \em 
enterprise management, 
Internet of Things, 
online social networks, 
medial applications, 
collective intelligence, and 
smart grid.
}

These discussions aim to provide a 
comprehensive overview and big-picture 
to readers of this exciting area. 

This survey is concluded with a discussion of 
open problems and future directions.

}

{
CAP Theory: Consistency; Availability; Partition Tolerance;

Eric Brewer proposed a CAP theory \cite{captheory} in 2000, 
which indicated that a distributed system could not simultaneously
meet the requirements on consistency, availability, and partition tolerance; 
at most two of the three requirements can be satisfied simultaneously. 

Seth Gilbert and Nancy Lynch from MIT proved the correctness of CAP theory \cite{capproof} in 2002. 

Since consistency, availability, and partition tolerance could not be achieved simultaneously, 
we can have 
a CA system by ignoring partition tolerance, 
a CP system by ignoring availability, and 
an AP system that ignores consistency,
according to different design goals. 

}

\bibliography{bd}
\end{document}










